\documentclass{article}

% if you need to pass options to natbib, use, e.g.:
% \PassOptionsToPackage{numbers, compress}{natbib}
% before loading nips_2016
%
% to avoid loading the natbib package, add option nonatbib:
% \usepackage[nonatbib]{nips_2016}

\usepackage[final]{nips_2016}

% to compile a camera-ready version, add the [final] option, e.g.:
% \usepackage[final]{nips_2016}

\usepackage[utf8]{inputenc} % allow utf-8 input
\usepackage[T1]{fontenc}    % use 8-bit T1 fonts
\usepackage{hyperref}       % hyperlinks
\usepackage{url}            % simple URL typesetting
\usepackage{booktabs}       % professional-quality tables
\usepackage{amsfonts}       % blackboard math symbols
\usepackage{nicefrac}       % compact symbols for 1/2, etc.
\usepackage{microtype}      % microtypography
\usepackage{color}


\newcommand{\selfnote}[1]{\footnote{\textcolor{red}{#1}}}

\title{Commuter classification and behavior clustering: Beijing use case}

% The \author macro works with any number of authors. There are two
% commands used to separate the names and addresses of multiple
% authors: \And and \AND.
%
% Using \And between authors leaves it to LaTeX to determine where to
% break the lines. Using \AND forces a line break at that point. So,
% if LaTeX puts 3 of 4 authors names on the first line, and the last
% on the second line, try using \AND instead of \And before the third
% author name.

\author{
  Selene Baez  Santamaria \\
  \texttt{s.baezsantamaria@student.vu.nl}
}

\begin{document}
% \nipsfinalcopy is no longer used

\maketitle

\begin{abstract}
  Public transportation, centered on subway and bus networks, is an data-rich domain that can benefit from data mining and machine learning techniques. The classification of commuters versus non-commuter/occasional travelers can help government, transport management and operators to better target their policies in order to improve the transportation network in large cities. Furthermore, characterizing commuters by behavior clustering can bring deeper insight into their needs and routines as a whole. 
  This project proposes the usage of ensemble models for classification and clustering of public transport users. For this purpose, transit card data will be used, available from the city of Beijing, China. 
\end{abstract}

\section{Introduction}

\subsection{Transportation domain}
Public transportation facilities. Common problem to identify regular users. 

\subsection{Beijing}
Resources of public tranpsortation network. Number of subway lines and bus lines. Number of users per day.

\subsection{Societal context}
Commuters use the public transport network regularly to go to work, school or other follow other routines. They need reliable means of transportation.

\subsection{Scientific context}
Usage of machine learning of data mining has been limited

Interdisciplinary study between Artificial Intelligence and Metropolitan Transportation.


\section{Related work}
Preprocess data by Wang in BJUT lab.

Machine learning for commuters identification. SVM with 94\% accuracy.

Ensemble methods

Classifiers in the transportation domain

\section{Objective}
Objective is to identify and characterize commuters in the city of Beijing by using IC card data.

\subsection{Research questions}
\begin{enumerate}
\item How accurately can commuters and non-commuters be identified using an ensemble model? How does this compare to the previous SVM model?
\item What is the minimal set of information needed from IC card data to reach an acceptable accuracy in classification?
\item To what extent is clustering commuters by its behavior informative to transportation specialists? 
\end{enumerate}

\section{Methodology}
\subsection{Data description}
Every record for an IC card contains the following data fields:

\begin{itemize}
\item Data date: date that the trip was made
\item Card code: card identification number
\item Data link: \selfnote{irrelevant?}
\item Path link: Mode of transportation. B for bus, R for subway. Transfers shown by a dash. Example: B-B is Bus to Bus.
\item Travel time: time spent in vehicles, measured in milliseconds
\item Travel distance: measured in meters \footnote{\textcolor{red}{as measured by route?}}
\item Transfer number: number of transfers in the trip
\item Transfer time average: time spent in transfer, divided by number of transfers
\item Start time: time stamp of when the trip started
\item End time: time stamp of when the trip ended
\item On traffic:
\item Off traffic:
\item On middle area:
\item Off middle area:
\item On big area:
\item Off big area:
\item On ring road:
\item Off ring road:
\item On area:
\item Off area:
\item ID: number| time stamp of beginning of trip | card code
\item Transfer detail: Station name, line number, mode of transportation
\end{itemize}

Every day, more than 50,000 records are collected. This project aims to include data from at least one week. 

\subsubsection{Training data}
Since we perform supervised learning, we need training data for which we know if a record corresponds to a commuter or non-commuter. Such data is expensive and limited since it has only been obtained by asking the users directly if they are commuters or not. Other annotated data is not available, and labeling new records falls beyond the scope of this project. \selfnote{if data is not sufficient (although previous work shows it is) I might need to consider annotating some data myself}

The current training and validation set consists of data from 2015, collected and validated by Tu[1] \selfnote{change references}. The data is composed by:

\begin{itemize}
\item 6439 records of 481 commuters
\item 1628 records of 497 non-commuters
\end{itemize}

For a total of 978 IC card IDs.

\subsubsection{Testing data}
Testing data

\subsection{Data cleaning}
Eliminate records that do not make sense or are faulty, for example having empty fields. 

\subsection{Redundant variables}
Hypothesis: middle area, big area and area overlap. Middle has more precision but maybe not needed.

\section{Results}
\subsection{Commuters identification}
Accuracy

Confusion matrix

\subsection{Variable evaluation}
\subsubsection{Qualitative}
Exploration: Experts opinion

\subsubsection{Quantitative}
Analysis: Correlation

\subsection{Commuters clustering}
\subsubsection{Expert judgment}

\section{Conclusion}

\section{Future work}

\section*{References}

\small

[1] Tu, Q.\ Weng, J. C.\ \& Yuan, R. L. Impact Analysis of Public Transport Fare Adjustment on Travel Mode Choice for Travelers in Beijing. {\it CICTP 2016.}, pp.\ 850--863.

\end{document}
