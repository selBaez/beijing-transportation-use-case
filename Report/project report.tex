\documentclass{article}

% if you need to pass options to natbib, use, e.g.:
% \PassOptionsToPackage{numbers, compress}{natbib}
% before loading nips_2016
%
% to avoid loading the natbib package, add option nonatbib:
% \usepackage[nonatbib]{nips_2016}

\usepackage[final]{nips_2016}

% to compile a camera-ready version, add the [final] option, e.g.:
% \usepackage[final]{nips_2016}

\usepackage[utf8]{inputenc} % allow utf-8 input
\usepackage[T1]{fontenc}    % use 8-bit T1 fonts
\usepackage{hyperref}       % hyperlinks
\usepackage{url}            % simple URL typesetting
\usepackage{booktabs}       % professional-quality tables
\usepackage{amsfonts}       % blackboard math symbols
\usepackage{nicefrac}       % compact symbols for 1/2, etc.
\usepackage{microtype}      % microtypography

\title{Commuter classification and behavior clustering: Beijing use case}

% The \author macro works with any number of authors. There are two
% commands used to separate the names and addresses of multiple
% authors: \And and \AND.
%
% Using \And between authors leaves it to LaTeX to determine where to
% break the lines. Using \AND forces a line break at that point. So,
% if LaTeX puts 3 of 4 authors names on the first line, and the last
% on the second line, try using \AND instead of \And before the third
% author name.

\author{
  Selene Baez  Santamaria \\
  \texttt{s.baezsantamaria@student.vu.nl}
}

\begin{document}
% \nipsfinalcopy is no longer used

\maketitle

\begin{abstract}
  Public transportation, centered on subway and bus networks, is an data-rich domain that can benefit from data mining and machine learning techniques. The classification of commuters versus non-commuter/occasional travelers can help government, transport management and operators to better target their policies in order to improve the transportation network in large cities. Furthermore, characterizing commuters by behavior clustering can bring deeper insight into their needs and routines as a whole. 
  This project proposes the usage of ensemble models for classification and clustering of public transport users. For this purpose, transit card data will be used, available from the city of Beijing, China. 
\end{abstract}

\section{Introduction}

\subsection{Transportation domain}

\subsection{Beijing}

\subsection{Societal context}
Public transport users

\subsection{Scientific context}
Usage of machine learning of data mining has been limited

Interdisciplinary study


\section{Related work}
Preprocess data 

Machine learning for commuters identification

Ensemble methods

Classifiers in the transportation domain

\section{Objective}
\subsection{Research questions}

\section{Methodology}

\section{Results}
\subsection{Commuters identification}
Accuracy

Confusion matrix

\subsection{Variable evaluation}
\subsubsection{Qualitative}
Exploration: Experts opinion

\subsubsection{Quantitative}
Analysis: Correlation

\subsection{Commuters clustering}
\subsubsection{Expert judgment}

\section{Conclusion}

\section{Future work}

\section*{References}

\small

[1] Tu, Q.\ Weng, J. C.\ \& Yuan, R. L. Impact Analysis of Public Transport Fare Adjustment on Travel Mode Choice for Travelers in Beijing. {\it CICTP 2016.}, pp.\ 850--863.

\end{document}
